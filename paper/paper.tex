%----------------------------------------------------------------------------------------
%	CONFIGURATIE VAN HET DOCUMENT
%----------------------------------------------------------------------------------------

\documentclass{article}

\usepackage{graphicx} 					% Voor het gebruik van afbeeldingen
\usepackage[square,numbers]{natbib} 	% Voor de bibliografie aan het eind
\usepackage{amsmath} 					% Nodig voor het gebruik van bepaalde wiskunde-elementen
\usepackage{wrapfig}					% Nodig voor het 'wrappen' van de tekst om de afbeelding heen
\usepackage{algorithmic}				% Voor het schrijven van pseudo-code
\usepackage[]{algorithm2e}				% "
\usepackage{courier}					% Voor het schrijven van code in de tekst

\setlength\parindent{0pt} 				% Haalt indentaties van paragrafen weg

\begin{document}

%----------------------------------------------------------------------------------------
%	TITELPAGINA
%----------------------------------------------------------------------------------------

\begin{titlepage}
		\newcommand{\HRule}{\rule{\linewidth}{0.5mm}}
		\center 						% Centreer alles op de pagina

		%----------------------------------------------------------------------------------------
		%	HEADER SECTIE
		%----------------------------------------------------------------------------------------

		\textsc{\large Universiteit Twente}\\[1.5cm]
		Module 7 - 201400433\\
		\vspace{5mm}
		\textsc{\Large Discrete Structuren \& Effici\"ente Algoritmes}\\[0.5cm]
		\vspace{15mm}

		%----------------------------------------------------------------------------------------
		%	TITEL SECTIE
		%----------------------------------------------------------------------------------------

		\HRule \\[0.4cm]
		{ \huge \bfseries Preprocessing methoden}\\[0.4cm]
		{ \large \bfseries Het effici\"ent maken van het Graaf Isormorfisme probleem}\\[0.4cm]
		\HRule \\[1.5cm]
		 
		%----------------------------------------------------------------------------------------
		%	AUTEUR(S) SECTIE
		%----------------------------------------------------------------------------------------

		Projectgroep 1\\
		\vspace{10mm}
		\begin{minipage}{0.4\textwidth}
		\begin{flushleft} \large
		\emph{Auteurs:}\\
		Joeri Kock\\
		Remco Brunsveld\\
		Frank Bruggink\\
		Dani\"el Schut
		\end{flushleft}
		\end{minipage}
		~
		\begin{minipage}{0.4\textwidth}
		\begin{flushright} \large
		\emph{Projectbegeleider:} \\
		Prof. Dr. J.C. van de Pol
		\end{flushright}
		\end{minipage}\\[4cm]

		%----------------------------------------------------------------------------------------
		%	DATUM SECTIE
		%----------------------------------------------------------------------------------------

		{\large \today}\\[3cm]

		\pagebreak

\end{titlepage}

%----------------------------------------------------------------------------------------
%	INHOUDSOPGAVE
%----------------------------------------------------------------------------------------

\renewcommand*\contentsname{Inhoudsopgave}
\tableofcontents{}
\pagebreak

%----------------------------------------------------------------------------------------
%	SECTIE 1: INLEIDING
%----------------------------------------------------------------------------------------

\section{Inleiding}
Dit paper zal gaan over het Graaf Isomorfisme probleem, en hoe dit effici\"ent gemaakt kan worden door middel van preprocessing. Het Graaf Isomorfisme probleem is een bekend probleem in de wiskunde en de informatica, wat het belangrijk maakt om dit op een zo snel mogelijke manier op te lossen. Als eerste wordt hierover (en over het algoritme wat wij hiervoor gebruiken) het een en ander uitgelegd. We zullen ook verschillende manieren van preprocessing bespreken, en toelichten waarom deze methoden geldig zijn binnen dit probleem. Daarna zullen we door middel van een experiment testen of deze methoden werken en wat het effect hiervan is op de snelheid en effici\"entie van ons algoritme. We sluiten dit document af met de conclusie.\\

We hebben in dit paper als hoofdvraag: ``In hoeverre heeft het gebruik van preprocessing invloed op de tijdsduur van het vinden van isomorfe grafen met het Color Refinement-algoritme?''. We hopen natuurlijk dat het gebruik van preprocessing van positieve invloed is op de tijdsduur, d.w.z. dat het geheel minder lang duurt.
\pagebreak

%----------------------------------------------------------------------------------------
%	SECTIE 2: HET GI PROBLEEM
%----------------------------------------------------------------------------------------

\section{Het Graaf Isomorfisme (GI) probleem}
Een graaf bestaat uit een verzameling punten, vertices genoemd, waarvan sommige verbonden zijn door lijnen, de edges. Afhankelijk van de toepassing kunnen de lijnen gericht zijn, dan worden ze ook wel pijlen genoemd, men spreekt dan van een gerichte graaf. Ook worden wel gewichten aan de lijnen toegekend door middel van getallen, deze stellen dan bijvoorbeeld de afstand tussen twee punten voor. Een graaf met gewichten noemt men een gewogen graaf. Wanneer we in dit paper een graaf noemen, hebben we het over een ongewogen ongerichte enkelzijdige graaf.\\

Een isomorfisme is een bijectieve afbeelding $f:V(G) \rightarrow V(H)$ zodanig dat $\forall u,v \in V(G)$ er geldt dat $uv \in E(G) \Leftrightarrow f(u)f(v)\in E(H)$.\\

Het Graaf Isomorfisme probleem is het probleem dat gaat over het vaststellen of twee grafen isomorf zijn, dat wil zeggen structureel gelijk aan elkaar. Als twee wiskundige objecten isomorf zijn, dan is elke eigenschap, waarvan de structuur bewaard blijft door een isomorfisme en die geldt voor een van de twee wiskundige objecten, ook geldig voor het andere wiskundige object.\\

\begin{wrapfigure}{r}{0.5\textwidth}
\renewcommand{\figurename}{Figuur}
\begin{center}
\includegraphics[width=0.48\textwidth]{img/isomorphic.png}
\end{center}
\caption{Twee isomorfe grafen}
\end{wrapfigure}

Hoewel het op het eerste gezicht niet zo lijkt te zijn, geldt er een isomorfisme tussen de twee grafen hiernaast. Voor elke vertex van de eerste graaf geldt dat de buren daarvan exact hetzelfde zijn als de buren van diezelfde vertex in de tweede graaf.\\

Een methode om isomorfisme vast te kunnen stellen tussen deze twee grafen is ‘color refinement’. Dit algoritme kent kleuren toe aan de verschillende vertices, die een indicatie geeft welke buren deze vertex heeft. De groene vertex bijvoorbeeld heeft in de eerste graaf drie buren: de lichtblauwe, de roze en de gele. Dit geldt ook voor de tweede graaf. Als dit voor elke vertex in de twee grafen geldt \`en elke vertex heeft een unieke kleur, zijn de twee grafen isomorf. Dit kan als volgt gecontroleerd worden: wanneer de lijst met alle kleuren (zonder duplicaten) van de eerste graaf gelijk is aan die van de tweede graaf, kan er gezegd worden dat ze isomorf zijn ten opzichte van elkaar. In de afbeelding hierboven is dat ook het geval.
Dit is een goede manier om isomorfisme te controleren, omdat het op het eerste gezicht bij deze afbeelding lijkt alsof de twee grafen niet isomorf zijn. Ook is deze methode nuttig bij grotere grafen met veel vertices.

\subsection{Relevantie van het GI probleem}
Nu duidelijk gemaakt is wat het GI probleem inhoudt en hoe het gecontroleerd kan worden, blijft de vraag hoe relevant dit probleem is voor de Discrete Wiskunde. Waarom willen we weten of twee grafen (of zelfs wiskundige problemen in het algemeen) isomorf zijn ten opzichte van elkaar?\\

Isomorfisme heeft al voor het oplossen van vele wiskundige problemen gezorgd. Dat kan als volgt gedaan worden: een nog op te lossen wiskundig probleem X kan herleid worden tot een eenvoudiger probleem Y, wat makkelijker te begrijpen is. Neem een reeds opgelost probleem Z. Als aangetoond kan worden dat Y en Z isomorf zijn ten opzichte van elkaar, kan de conclusie getrokken worden dat op dat moment probleem X ook opgelost is (omdat probleem X, Y en Z structureel hetzelfde zijn). In de Discrete Wiskunde zijn er al veel problemen met deze methode opgelost, wat het controleren op isomorfisme erg relevant maakt.
Naast de wiskunde is het ook nuttig voor de informatica, omdat daar veel structuren als grafen weergegeven worden (denk aan netwerksystemen).\\

Een goed voorbeeld hiervan in de informatica zijn vingerafdrukscanners. Wanneer een computer een vingerafdruk scant, wordt deze versimpeld naar enkele lijnen. Elke lijn representeert een vertex in de graaf, en elke edge representeert een relatie in de omgeving van deze lijn(en). Met deze techniek wordt een vingerafdruk versimpeld naar een graaf met vertices en edges. Als men wil controleren of twee vingerafdrukken gelijk zijn aan elkaar, moet men een isomorfisme bewijzen tussen de twee grafen die bij die vingerafdrukken horen.

\subsection{Speciale gevallen}
Er is een aantal speciale gevallen waarvan bewezen is dat het in polynomiale tijd opgelost kan worden. Het eerste geval is een graaf die een boomstructuur heeft, dat wil zeggen: elk punt is met een ander punt verbonden, zonder cycli.\\
De tweede is een planaire graaf, dat is een graaf die zo in het platte vlak getekend kan worden dat geen van de randen een andere rand kruist. Ook als je een restrictie legt op het aantal keren dat de randen elkaar minimaal kruisen, of als je een restrictie legt op de maximale graad van alle hoekpunten is het in polynomiale tijd op te lossen.
\pagebreak

%----------------------------------------------------------------------------------------
%	SECTIE 3: HET ALGORITME VOOR COLOR REFINEMENT
%----------------------------------------------------------------------------------------

\section{Het algoritme voor Color Refinement}
Hier zullen we een korte uitleg geven over de implementatie van ons color refinement-algoritme.\\



Voordat we color refinement toepassen, geven we eerst elke vertex van de graaf een kleur die gelijk staat aan zijn degree. Het kleuren gebeurt niet in de graaf zelf, maar wordt apart bijgehouden door 2 arrays. De eerste array houdt voor elke kleur bij welke vertices die kleur hebben. De tweede array houdt voor elke vertex bij welke kleur hij heeft. Op deze manier hebben we de graaf zelf (bijna) niet meer nodig en kunnen we alle bewerkingen op de twee arrays uitvoeren, wat het proces sneller en overzichtelijker maakt.\\
Na de initi\"ele kleuring voeren we color refinement uit over de lijsten. Als blijkt dat er nu geen duplicate kleuren meer zijn binnen elke graaf (d.w.z. binnen een graaf bestaan er geen twee vertices met dezelfde kleur) is het klaar en kunnen we controleren welke graven isomorf zijn. Zijn er wel duplicate kleuren, dan moet `individual refiment' toegepast worden. De pseudo-code voor het eerste deel staat hieronder:\\

\begin{algorithm}[H]
	\KwData{Een lijst met initiele kleuring van de vertices en een lijst met vertices per kleur}
	\KwResult{Een lijst met stabiele kleuring van de vertices en een lijst met vertices per kleur}
	\For{Alle vertices v}{
		\For{Alle vertices k die de kleur van v hebben}{
			nieuweKleuren = []\\
			\If{De kleuren van de neighbors van v ongelijk zijn aan de neigbors van k}{
				\For{Alle elementen q in nieuweKleuren}{
					\eIf{De neighbors van k gelijk zijn aan die van q}{
						Kleur van k = q\\
					}{	Maak een nieuwe kleur voor k aan\\
						Voeg de kleur toe aan de lijst nieuweKleuren\\
					}
				}
			}
		}
	}
	\If{De nieuwe lijsten ongelijk zijn aan de oude lijsten}{
		Recursie van het algoritme met nieuwe lijsten als data\\
	}
	\Return De twee aangepaste lijsten
\end{algorithm}
\vspace{5mm}

Individual refinement wordt alleen toegepast op de graven met duplicate kleuren. Het algoritme werkt als volgt: hij zoekt eerst een kleur op die bezet wordt door meerdere vertices. Vervolgens neemt hij \'e\'en van deze vertices, en vervangt zijn kleur. Bij beide grafen wordt hiervoor een andere (nieuwe) kleur gekozen.Hierna wordt color refinement opnieuw toegepast. Als er nog steeds duplicate kleuren bestaan wordt color refinement nogmaals toegepast. Als er een isomorfisme aangetoond kan worden, wordt individual refinement over een ander koppel graven toegepast. Is er geen isomorfisme dan wordt de herkleuring van graaf 1 en graaf 2 teruggedraaid en wordt er een andere duplicate kleur van graaf 2 vervangen, waarna het algoritme zich weer herhaalt.\\

\begin{algorithm}[H]
	\KwData{Een lijst met stabiele kleuring van de vertices en een lijst met vertices per kleur}
	\KwResult{Een lijst met unieke kleuring van de vertices per graaf en een lijst met vertices per kleur}
	dupVertices = dictionary met alle vertices van een dubbele kleur per graaf\\
	\For{Alle graven g in dupVertices}{
		\For{Alle vertices v van g}{
			Verander de eerste kleur van de eerste graaf in dupVertices in een nieuwe kleur\\
			Verander de kleur van v in een nieuwe kleur\\
			Pas colorrefinement toe op de twee lijsten\\
			\If{De kleuren van de twee aangepaste grafen zijn gelijk}{
				\eIf{Er zijn geen grafen zonder unieke kleuring meer}{
					\Return de nieuwe lijsten
				}{	Pas individual refinement opnieuw toe\\
					\If{Er is geen isomorfisme gevonden}{
						Zet de staat van de lijsten terug naar het origineel
					}
				}
			}
		}
	}
	\Return De lijsten met unieke kleuring
\end{algorithm}
\pagebreak

%----------------------------------------------------------------------------------------
%	SECTIE 4: PREPROCESSING
%----------------------------------------------------------------------------------------

\section{Preprocessing}
Om het algoritme een handje te helpen bij het effici\"ent oplossen van het GI probleem, kunnen er voor het color refinement-algoritme al stappen ondernomen worden, zodat het algoritme zelf minder tijd nodig heeft en dus het algoritme sneller en effici\"enter wordt.
Voor het algoritme kunnen er namelijk grafen ge\"elimineerd worden uit de lijst die niet in aanmerking komen om isomorf te zijn met een andere graaf uit de lijst. Als je van tevoren deze grafen uit de lijst haalt, hoeft het color refinement-algoritme minder grafen te behandelen, waardoor het effici\"enter wordt.

Voor het testen gebruik je een lijst van grafen, bijvoorbeeld een lijst met 4 grafen. Als er \'e\'en graaf is in de lijst die een van de eigenschappen van het preprocessing niet heeft en de andere drie wel, kan deze graaf uit de lijst gehaald worden. Het is dan namelijk niet meer mogelijk dat deze graaf isomorf is met \'e\'en van de andere grafen. Maar als twee grafen de eigenschappen wel hebben en twee niet, zetten we deze paren van grafen in twee aparte lijsten, waar vervolgens het color refinement-algoritme op uitgevoerd kan worden.

\subsection{Methoden}
We controleren de grafen van tevoren op de volgende eigenschappen:
\subsubsection{Grootte}
De grootte van de graaf, d.w.z. het aantal vertices en het aantal edges. Als er in de lijst één graaf is met een ander aantal vertices of edges dan de andere grafen, kan deze graaf uit de lijst gehaald worden. Hij kan dan niet meer isomorf zijn met een andere graaf. De pseudo-code voor het testen op grootte staat hieronder:\\

\begin{algorithm}[H]
	\KwData{Een lijst van grafen}
	\KwResult{Een lijst van grafen die mogelijk isomorf zijn}
	resultaat = lege lijst\\
	\For{Elke graaf i in de lijst}{
		\For{Elke andere graaf j in de lijst}{
			\If{Het aantal vertices van graaf i en j zijn gelijk}{
				\If{Het aantal edges van graaf i en j zijn gelijk}{
					Voeg i en j toe aan het resultaat
				}
			}
		}
	}
	\Return resultaat
\end{algorithm}

\subsubsection{Connectiviteit}
Als een graaf als enige in de lijst niet ‘connected’ is (en de andere grafen dus wel), kan deze ook ge\"elimineerd worden. Met een combinatie van deze eerste twee eigenschappen wordt ook gelijk gecontroleerd op het feit of een graaf een boom is of niet. Als twee ‘connected’ grafen een gelijk aantal vertices en edges hebben, zijn ze namelijk beide een boom of beide niet. Hier hoeft dus niet meer op gecontroleerd te worden. De pseudo-code voor het testen op connectiviteit staat hieronder:\\

\begin{algorithm}[H]
	\KwData{Een lijst van grafen}
	\KwResult{Een lijst van grafen die mogelijk isomorf zijn}
	resultaat = lege lijst\\
	\For{Elke graaf i in de lijst}{
		\For{Elke andere graaf j in de lijst}{
			\If{Grafen i en j zijn beide connected}{
				Voeg i en j toe aan het resultaat
			}
		}
	}
	\Return resultaat
\end{algorithm}
\vspace{5mm}

In de If-statement staat dat we controleren of beide grafen i en j connected zijn. We hebben een algoritme geschreven (dat je op een graaf kunt aanroepen) dat teruggeeft of een graaf connected is of niet. De pseudo-code hiervan staat hieronder beschreven:\\

\begin{algorithm}[H]
	\KwData{De graaf G}
	\KwResult{True als hij connected is, anders False}
	x = een willekeurige vertex\\
	L = een lijst van vertices bereikbaar vanaf x\\
	K = een lijst van vertices die nog ontdekt moeten worden\\
	Aan het begin van het algoritme geldt L = K = x\\

	\While{K is niet leeg}{
		Vind een vertex y in K en haal deze eruit\\
		\For{elke edge (y,z)}{
			\If{z zit niet in L}{
				Voeg z toe aan L en K\\
			}
		}
	}
	\eIf{L heeft minder elementen dan het aantal vertices van G}{
		\Return False
	}{\Return True}

\end{algorithm}
\vspace{5mm}

Nog een extra mogelijkheid bij het controleren op connectiviteit is het splitsen van grafen. Als een graaf niet connected is, kan deze opgesplitst worden in een connected deel en de rest van de graaf. Het eerste deel is nu wel connected, en kan om die reden dus vergeleken worden met andere connected grafen in de lijst (mits het aantal vertices en edges gelijk zijn aan die van de andere grafen). Helaas is het niet gelukt om deze functionaliteit te implementeren in onze code.

\subsubsection{False twins}
Een andere mogelijkheid is het controleren of een graaf `False Twins' heeft. False twins zijn twee vertices die exact dezelfde buren hebben, maar niet met elkaar verbonden zijn. Als het aantal false twins in graaf G ongelijk is aan het aantal false twins in graaf H, kunnen G en H niet meer isomorf zijn. Helaas is het hier ook niet gelukt om deze methode binnen de tijd te implementeren, maar het bewijs voor de validatie is wel in de volgende sectie te vinden.

\subsubsection{Modules}
Een module is een groep vertices die buiten de module dezelfde buren hebben. Als een vertex X (buiten de module) verbonden is met een vertex in een bepaalde module, zijn alle vertices in die module verbonden met X. Natuurlijk is elke vertex zelf ook een module, net als de gehele graaf. Als graaf G een ongelijk aantal modules heeft aan graaf H, kunnen G en H niet meer isomorf zijn. Dit hebben we niet ge\"implementeerd in onze code en ook niet bewezen in de volgende sectie (vanwege de tijd).

\subsection{Validatie preprocessing methoden}

Een isomorfisme is een bijectieve afbeelding $f:V(G) \rightarrow V(H)$ zodanig dat $\forall u,v \in V(G)$ er geldt dat $uv \in E(G) \Leftrightarrow f(u)f(v)\in E(H)$.\\

We hebben 3 preprocessing stappen gemaakt.
\begin{enumerate}
\item Als $|V(G)|\neq|V(H)|$ dan kan er geen isomorfisme bestaan.
\item Als $|E(G)|\neq|E(H)|$ dan kan er geen isomorfisme bestaan.
\item Als een graaf wel `connected' is en de andere niet, dan kan er geen isomorfisme bestaan.
\item Als het aantal `false twins' van $G$ een $H$ niet gelijk is, kan er geen isomorfisme bestaan.
\end{enumerate}

Nu volgt het bewijs dat twee grafen niet isomorf zijn als aan een van de volgende 3 voorwaarden wordt voldaan:
\begin{enumerate}
\item Als $|V(G)|\neq|V(H)|$ dan kan je geen bijectie defini\"eren, want een bijectie moet one-to-one zijn, en daarvoor moeten beide sets evenveel elementen bevatten.
\item Als $|E(G)|\neq|E(H)|$ dan is er een tegenspraak met de eis dat $uv\in E(G)\Leftrightarrow f(u)f(v)\in E(H)$.
\item Neem aan dat $H$ connected is en $G$ niet. Zonder verlies van algemeenheid kunnen we aannemen dat $G$ bestaat uit twee delen die beide connected zijn, noem ze $G_1$ en $G_2$.
\item Twee vertices $u, v \in V(G)$ zijn false twins als $N(u)=N(v)$. Als graaf $G$ en graaf $H$ niet hetzelfde aantal false twins hebben, dan kan er geen isomorfisme bestaan tussen $G$ en $H$, omdat een isomorfisme de edges behoudt, en dus ook de neighborhood.

\vspace{5mm}
Neem nu aan dat er een isomorfisme tussen $G$ en $H$ bestaat. Deel $H$ op in twee delen, $H_1$ en $H_2$, zodanig dat $\forall g_1\in V(G_1)$ geldt dat $f(g_1)\in V(H_1)$ en $\forall g_2\in V(G_2)$ geldt dat $f(g_2)\in V(H_2)$.
Dan $\exists g_1\in G_1$ en $g_2\in G_2$ zodanig dat $g_1g_2\in E(G)$, terwijl $f(g_1)f(g_2)\notin E(H)$. Dit levert een tegenspraak op.\\
Dus de grafen moeten beide wel of beide niet connected zijn.
\end{enumerate}

\pagebreak

%----------------------------------------------------------------------------------------
%	SECTIE 5: TESTEN
%----------------------------------------------------------------------------------------

\section{Testen}
We hebben onze code uitvoerig getest op verschillende delen. Als eerste hebben we testklassen gemaakt in Python die uitgevoerd kunnen worden en (met behulp van wat randgevallen) testen of onze preprocessing-methoden naar behoren werken. Dit zijn zogenaamde `unittests' die als volgt werken: we testen eerst of onze functie \texttt{isConnected} werkt. Dit doen we door deze functie uit te voeren op vier grafen, waarvan er \'e\'en niet connected is. Hier voeren we dus een drietal \texttt{assertTrue}- en \'e\'en \texttt{assertFalse}-functie(s) op uit.\\
Bovendien kijken we of het aanpassen van de lijsten werkt. Immers, als er \'e\'en graaf als enige niet connected is of een ander aantal vertices of edges heeft, moet deze uit de lijst gehaald worden. We voeren hiervoor de betreffende methoden uit, en we controleren naderhand of de aangepaste lijst overeenkomt met de lijst die we voorspeld hadden.\\

We hebben naast de werking ook het effect van onze algoritmes getest. Met behulp van verschillende lijsten van grafen (waaronder we er een aantal zelf geschreven hebben) hebben we experimenten uitgevoerd waar we hebben gekeken of de preprocessing-methoden ook daadwerkelijk effect hadden op de tijd die het algoritme erover doet. Het resultaat van dit experiment is te zien in de grafiek hieronder.
\begin{center}
\includegraphics[width=1\textwidth]{img/test.png}
\end{center}
De waarden in de grafiek staan voor hoeveel sneller of langzamer het algoritme er procentueel over doet t.o.v. zonder de preprocessing. Wat erg duidelijk te zien is, is dat als we \'e\'en grafiek een ander aantal vertices geven, we een significante verbetering zien wanneer we hierop controleren d.m.v. preprocessing. Dit is bij de connectiviteit ook het geval. Wanneer de grafen al isomorf zijn, heeft het natuurlijk enkel een negatief effect. Je voert namelijk extra controles uit op eigenschappen waarvan je weet dat dit niet hoeft, aangezien de grafen al isomorf zijn.
\pagebreak

%----------------------------------------------------------------------------------------
%	SECTIE 6: CONCLUSIE
%----------------------------------------------------------------------------------------

\section{Conclusie}
Op basis van de gemeten testresultaten kunnen we een conclusie trekken over de gebruikte methoden van preprocessing. Dat is dat ze wel degelijk verschil maken in de uiteindelijke tijd, mits het ook noodzakelijk is deze controles uit te voeren. Als er in de lijst een graaf is die als enige connected is, wordt deze uit de lijst gehaald. Dit scheelt het Color Refinement-algoritme erg veel tijd. Heb je echter een lijst van grafen die allemaal evenveel vertices en edges hebben \`en allemaal connected of niet connected zijn, kost het algoritme een (klein) beetje extra tijd, omdat je controles uitvoert die niet nodig zijn.
\pagebreak

%----------------------------------------------------------------------------------------
%	REFERENTIES
%----------------------------------------------------------------------------------------

\renewcommand{\refname}{Referenties}
\bibliographystyle{amsplain}
\bibliography{paper}

\end{document}